\documentclass[11pt, a4paper]{article}

\usepackage{authblk}
\usepackage{hyperref}
\usepackage{glossaries}

\renewcommand{\baselinestretch}{1.0}

%
\newcommand*{\email}[1]{%
    \normalsize\href{mailto:#1}{#1}\par
    }

\author{Nguyen Hoang Anh}
\affil{John Abbott College, Sainte-Anne-de-Bellevue, Quebec\\ \email{hoanganh.theodore@icloud.com}}
\date{January 25, 2020}
\title{Applications of Built-in Accelerometers and Inertia Sensors in Modern Mobile Devices}



\begin{document}
    \maketitle

    \begin{center}
        \textbf{Proposal}

        Independent Research Project in Science 360-RES-AB       
    \end{center}
    \textit{Keywords:} accelerometer, inertia sensor, Python programming, accelerometric data processing, physics, computer science, mechanical motions, designing experiments, computational data collection

    \section{Introduction}
    While for some learning Physics is a bliss, for others it might be a living nightmare. Part of the problem can be attributed to the lack of real-world demonstrations of various mechanical phenomena throughout the learning curve. Thus, the learner feels disconnected towards the subject, which trickles down to unable to grasp important concepts. In a seemingly different aspect, modern smartphones are equipped with incredible tools and sensors that enable the user to perform most of their day-to-day tasks with ease, like communicating, navigating, and storing important data. However, there are more specialized ways to utilize such high-performing integrated tools than calling Uber or sending cat GIFs to a friend. Two seemingly distinct problems might just be the complementing DNA strands of their own respective solution. Modern day students may depend their whole education on the mobile devices that they carry daily, always within an arm's length, ready to use. Instead of having students put away their devices as a source of distraction in class, the research tries to discover more integrated ways in which these devices essentially aid the process of achieving the competency for their class; in the context of this research, the field of physics, more specifically, mechanics.
    \section{Methodology}
    With this in mind, our research team devises methods to implement phones in educating future students about classical mechanics. The most commonly known area of physics 
    HELP MEEEEEEEEEEEEEEEEEEEE!!!!!!!!!!!!!!!!!!!!!! REEEEEEEEEEEEEEEEEEEEEEEEEEEEEEEEEEEE!
    
    GIT WARS: THE CLONE WARS
\end{document}