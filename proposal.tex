\documentclass[11pt, a4paper]{article}

\usepackage{authblk}
\usepackage{hyperref}
\usepackage{glossaries}


\renewcommand{\baselinestretch}{1.0}

%
\newcommand*{\email}[1]{%
    \normalsize\href{mailto:#1}{#1}\par
    }

\author{Nguyen Hoang Anh}
\affil{John Abbott College, Sainte-Anne-de-Bellevue, Quebec\\ \email{hoanganh.theodore@icloud.com}}
\date{January 25, 2020}
\title{Analysis and Implementation of Machine Learning of Built-in Inertia Sensors in Modern Mobile Devices}



\begin{document}
    \maketitle

    \begin{center}
        \textbf{Proposal}

        Independent Research Project in Science 360-RES-AB       
    \end{center}
    \begin{center}
        \textbf{Advisor:}

        Prof. Chris Isaac Larnder (Physics Department)
    \end{center}
    \textit{Keywords:} accelerometer, inertia sensor, Python programming, accelerometric data processing, physics, computer science, mechanical motions, designing experiments, computational data collection

    \section{Background}
    Modern smartphones are equipped with incredible tools and sensors that enable the user to perform most of their day-to-day tasks with ease, like communicating, navigating, and storing important data. However, there are more specialized ways to utilize such high-performing integrated tools than calling Uber or sending cat GIFs to a friend. People may depend most of their daily activities on the mobile devices that they carry daily, always within an arm's length, ready to use. Instead of stigmatizing our smartphone addiction, this research tries to discover more useful ways that one can squeeze more mileage out of their personal device.
    \section{Description}
    With this in mind, our research team devises methods to locate the accelerometer in the devices by iterative data collections and holistic data analyses.
\end{document}