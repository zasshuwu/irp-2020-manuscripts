\documentclass{article}
\author{Nguyen Hoang Anh}
\date{January 25, 2020}
\title{Applications of Accelerometers and Inertia Sensors in Pedagogical and Experimental Settings}


\usepackage{indentfirst}
\usepackage[utf-8]{babel}

\begin{document}
    \maketitle

    \section*{Proposal}
    Take a stroll along a crowded streets in the evening and one cannot help but to notice how many people who are using their phones, whether to find a way around town or to schedule a meet up with friends. The phone has steadily become more and more personal and vital to an average joe's everyday life. The assuring feeling of having your phone with you, working without a hitch on whatever task you need it for, further solidifies the mobile device's status of importance in your life.

    The mobile phone has undergone great a many innovations and reimaginations. Ever since Apple first unveiled its first smartphone, iPhone Generation 1, in June 2007, we, collectively as end-users, have seen countless ways phonemakers do to improve the performance, convenience, reliability and utility of the phone. Most noticeable among such methods are equipping the devices with powerful chipsets (high-performance and energy-efficient CPUs\footnote{Central Processing Unit} and GPUs\footnote{Graphical Processing Unit}), incorporating advanced communication protocols (GPS\footnote{Global Positioning System}, Bluetooth, 5G/LTE\footnote{A speed-based data transfer standard}, cell tower triangulation), integrating a hostful of state-of-the-arts (SoTA) sensors, and etc. Aside from the apparent fact that these improvements serve to streamline the way we use our phones for daily tasks, we shall define them as the peripheral tasks, like navigation, online/offline media consumption, or communications, since the devices are equipped with abundant features, we can further our mileage with the devices by using the integrated sensors for their original purposes a.k.a. its primary tasks.

    As mentioned before, there are many sensors in the phones that we use day-to-day. With sensors that high of a caliber, we can use them as alternatives to specialty-made sensors that might cost a fortune to purchase for experimental data collection. In the case of our research, we focus on the accelerometers built into the phones. There are applications on the respective app stores that allow us to directly record metric data in real time and store them for later data analysis.

\end{document}